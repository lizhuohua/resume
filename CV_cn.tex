%----------------------------------------------------------------------------------------
%	PACKAGES AND OTHER DOCUMENT CONFIGURATIONS
%----------------------------------------------------------------------------------------

\documentclass[11pt,a4paper,roman]{moderncv} % Font sizes: 10, 11, or 12; paper sizes: a4paper, letterpaper, a5paper, legalpaper, executivepaper or landscape; font families: sans or roman

\moderncvstyle{banking} % CV theme - options include: 'casual' (default), 'classic', 'oldstyle' and 'banking'
\moderncvcolor{black} % CV color - options include: 'blue' (default), 'orange', 'green', 'red', 'purple', 'grey' and 'black'

\usepackage[scale=0.85]{geometry} % Reduce document margins
% \setlength{\hintscolumnwidth}{3cm} % Uncomment to change the width of the dates column
% \setlength{\makecvtitlenamewidth}{10cm} % For the 'classic' style, uncomment to adjust the width of the space allocated to your name
\usepackage[utf8]{inputenc}
\usepackage[T1]{fontenc}
% \usepackage{lmodern}
\usepackage{xeCJK} % Chinese font setting
\usepackage{microtype}
\usepackage{fontawesome5}

% Customize icon link
% usage: \iconlink[<icon:fontawesome>][<text>]{<url>}
\NewDocumentCommand{\iconlink}{ o o m }{
  \IfNoValueTF{#1}{
    \let\icon\faLink
  }{
    \let\icon#1
  }
  \IfNoValueTF{#2}{
    \hspace{-12pt}\underline{\href{#3}{\icon}}
  }{
    \hspace{-12pt}\underline{\href{#3}{\icon\ #2}}
  }
  \hspace{-8pt}
}
% \setCJKmainfont{Source Han Sans CN}
% \setCJKsansfont{Source Han Sans CN}

% \setCJKmainfont[
% BoldFont=Heiti SC,
% ItalicFont=KaiTi,
% SmallCapsFont=SimHei
% ]{SimSun}
% \setCJKsansfont{KaiTi}
% \setCJKmonofont{SimHei}

%----------------------------------------------------------------------------------------
%	NAME AND CONTACT INFORMATION SECTION
%----------------------------------------------------------------------------------------
\firstname{李} % Your first name
\familyname{卓华} % Your last name

% All information in this block is optional, comment out any lines you don't need
% \title{Curriculum Vitae}
\title{简历}
%\title{Resume}
% \address{The Chinese University of Hong Kong, Shatin, N.T., Hong Kong SAR}{}
\phone[mobile]{(+852)~6587~3796}
%\phone{(000) 111 1112}
%\fax{(000) 111 1113}
\email{zhli@cse.cuhk.edu.hk}
\homepage{zhuohua.me}
\social[github]{lizhuohua}
\social[linkedin][www.linkedin.com/in/lizhuohua]{Zhuohua Li}
\social[googlescholar][scholar.google.com/citations?user=h8XVAkgAAAAJ]{Zhuohua Li}
%\extrainfo{additional information}
%\photo[70pt][0.4pt]{pictures/picture} % The first bracket is the picture height, the second is the thickness of the frame around the picture (0pt for no frame)
%\quote{"A witty and playful quotation" - John Smith}

%----------------------------------------------------------------------------------------

\begin{document}

%----------------------------------------------------------------------------------------
%	CURRICULUM VITAE
%----------------------------------------------------------------------------------------
\vspace*{-5mm}
\makecvtitle % Print the CV title

%----------------------------------------------------------------------------------------
%	EDUCATION SECTION
%----------------------------------------------------------------------------------------
\vspace*{-10mm}
\section{教育经历}
\cventry{2017年8月--2022年10月}{博士,计算机科学与工程学系}{香港中文大学}{香港}{}{网络系统研究实验室(ANSR Lab)\\导师:John C.S. Lui教授}
\cventry{2013年8月--2017年6月}{学士,计算机科学与技术系}{中国科学技术大学}{合肥}{}{华夏计算机科技英才班}

%----------------------------------------------------------------------------------------
%	WORK EXPERIENCE SECTION
%----------------------------------------------------------------------------------------
\section{工作经历}
\cventry{2023年8月--现在}{导师:John C.S. Lui教授}{香港中文大学\ 博士后研究员}{香港}{}
{
  \cvlistitem{研究在线学习和多臂老虎机算法及其应用}
}
\cventry{2022年10月--2023年8月}{导师:John C.S. Lui教授}{香港中文大学\ 初级研究助理}{香港}{}
{
  \cvlistitem{设计适用于量子网络的边界网关协议 (Border Gateway Protocol, BGP)}
  \cvlistitem{使用在线学习算法选择高保真度的量子链路并减少测量消耗}
}
\cventry{2018年7月--2018年10月}{导师:韦韬\ 博士}{用Rust开发安全的Linux内核模块}{百度X-Lab,美国}{}
{
  \cvlistitem{开发了用Rust编程语言编写Linux内核模块的基础框架}
  \cvlistitem{提供了安全的内核态内存管理机制和内核锁同步机制}
}

\cventry{2016年7月--2016年9月}{社会计算组}{微软小冰的情感分析}{微软亚洲研究院,北京}{导师:谢幸\ 博士}
{
  \cvlistitem{参与改进微软小冰聊天机器人的情感分析能力}
  \cvlistitem{根据对话的上下文识别当前对话中用户的情感,并作出相应回答}
}

\cventry{2016年6月}{导师:安虹\ 教授}{ISC16国际大学生超级计算机竞赛}{法兰克福,德国}{}{
  \cvlistitem{基于华为FusionServer X6800服务器搭建超算集群,在3千瓦功耗限制下优化应用性能}
  \cvlistitem{获得总分第四名,WRF (Weather Research and Forecasting Model)单项第一名}
}

% ----------------------------------------------------------------------------------------
% PUBLICATIONS SECTION
% ----------------------------------------------------------------------------------------
\section{发表论文\textnormal{\normalsize (*表示共同第一作者,\#表示通讯作者)}}

\cvlistitem{Maoli Liu, \textbf{Zhuohua Li}\textsuperscript{\#}, Xuchuang Wang, and John C.S. Lui. \textsc{LinkSelFiE}: Link Selection and Fidelity Estimation in Quantum Networks. In \textit{Proceedings of the IEEE Conference on Computer Communications 2024}. INFOCOM '24.}

\cvlistitem{Maoli Liu\textsuperscript{*}, \textbf{Zhuohua Li}\textsuperscript{*\#}, Kechao Cai, Jonathan Allcock, Shengyu Zhang, and John C.S. Lui. Quantum BGP with Online Path Selection via Network Benchmarking. In \textit{Proceedings of the IEEE Conference on Computer Communications 2024}. INFOCOM '24.}

\cvlistitem{Jincheng Wang, \textbf{Zhuohua Li}, Mingshen Sun, Bin Yuan\textsuperscript{\#}, and John C.S. Lui. \href{https://doi.org/10.1109/DSN58367.2023.00053}{IoT Anomaly Detection Via Device Interaction Graph}. In \textit{Proceedings of the 53rd Annual IEEE/IFIP International Conference on Dependable Systems and Network}. DSN '23.}

\cvlistitem{\textbf{Zhuohua Li}, Jincheng Wang, Mingshen Sun, and John C.S. Lui. \href{https://doi.org/10.1007/978-3-031-17143-7_33}{Detecting Cross-Language Memory Management Issues in Rust}. In \textit{Proceedings of the 27th European Symposium on Research in Computer Security}. ESORICS '22.}

\cvlistitem{Jincheng Wang, \textbf{Zhuohua Li}, John C.S. Lui, and Mingshen Sun. \href{https://doi.org/10.1145/3545948.3545953}{Zigbee's Network Rejoin Procedure for IoT Systems: Vulnerabilities and Implications}. In \textit{Proceedings of the 25th International Symposium on Research in Attacks, Intrusions and Defenses}. RAID '22.}

\cvlistitem{Jincheng Wang, \textbf{Zhuohua Li}, John C.S. Lui, and Mingshen Sun. \href{https://doi.org/10.1016/j.cose.2021.102552}{Topology-Theoretic Approach To Address Attribute Linkage Attacks In Differential Privacy}. \textit{Computers \& Security}, Volume 113, February 2022.}

\cvlistitem{\textbf{Zhuohua Li}, Jincheng Wang, Mingshen Sun, and John C.S. Lui. \href{https://doi.org/10.1145/3460120.3484541}{\textsc{MirChecker}: Detecting Bugs in Rust Programs via Static Analysis}. In \textit{Proceedings of the 28th ACM Conference on Computer and Communications Security}. CCS '21.}

\cvlistitem{Jincheng Wang, \textbf{Zhuohua Li}, John C.S. Lui, and Mingshen Sun. \href{https://doi.org/10.1109/INFOCOMWKSHPS51825.2021.9484499}{Topology-Theoretic Approach To Address Attribute Linkage Attacks In Differential Privacy}. In \textit{Proceedings of IEEE INFOCOM WKSHPS: BigSecurity 2021: International Workshop on Security and Privacy in Big Data}. BigSecurity '21.}

\cvlistitem{\textbf{Zhuohua Li}, Jincheng Wang, Mingshen Sun, and John C.S. Lui. \href{https://doi.org/10.1145/3339252.3340506}{Securing the Device Drivers of Your Embedded Systems: Framework and Prototype}. In \textit{Proceedings of the 3rd International Workshop on Security and Forensics of IoT (in conjunction with ARES 2019)}. IoT-SECFOR '19.}

%----------------------------------------------------------------------------------------
%	Projects SECTION
%----------------------------------------------------------------------------------------
\section{项目经历}
\cvitemwithcomment{\textsc{FFIChecker} (\faIcon[regular]{star}35)}{Rust跨语言场景下的内存管理错误检测工具}{\iconlink[\faGithub]{https://github.com/lizhuohua/rust-ffi-checker}}
  \cvlistitem{检测Rust中由外部函数接口\textit{(Foreign Function Interface)}导致的内存管理错误}
  \cvlistitem{检测到34个现实中的漏洞}

\cvitemwithcomment{\textsc{MirChecker} (\faIcon[regular]{star}111)}{Rust程序的静态分析与错误检测工具}{\iconlink[\faGithub]{https://github.com/lizhuohua/rust-mir-checker}}
  \cvlistitem{基于\textit{抽象释义}对Rust的\textit{中级中间表示}(\textit{Mid-level Intermediate Representation})进行静态分析}
  \cvlistitem{检测到33个现实中的运行时崩溃(Panic)和内存安全漏洞}

\cvitemwithcomment{linux-kernel-module-rust (\faIcon[regular]{star}630)}{安全的Linux内核模块Rust开发框架}{\iconlink[\faGithub]{https://github.com/lizhuohua/linux-kernel-module-rust}}
  \cvlistitem{开发了用Rust编程语言编写Linux内核模块的基础框架}
  \cvlistitem{提供了安全的内核态内存管理机制和内核锁同步机制}

% \cventry{2015年9月--2016年2月}{导师:张煜\ 副教授}{基于LLVM的简易C语言编译器}{中国科学技术大学}{}
% {
% 	\begin{itemize}
% 		\item 使用\emph{Flex}+\emph{Bison}+\emph{LLVM}实现的简易C语言编译器
% 		\item 支持Clang风格的报错信息
% 		\item 支持编译基于\textit{Message Passing Interface (MPI)}的并行程序
% 	\end{itemize}
% }

% \cventry{2015年3月--2015年7月}{导师:李希\ 副教授}{基于FPGA实现的MIPS16e处理器}{中国科学技术大学}{}
% {
% 	\begin{itemize}
%     \item 在\emph{Xilinx FPGA}上使用\emph{Verilog HDL}语言实现,支持全部\emph{MIPS16e}指令
% 		\item 通过\emph{RS-232}串口实现DEBUG模块,可打印DEBUG信息到屏幕 
% 	\end{itemize}
% }

% \cventry{2015年3月--2015年6月}{导师:邢凯\ 副教授}{移植Inferno操作系统至三星S3C6410开发板}{中国科学技术大学}{}
% {
% 	\begin{itemize}
% 		\item 使用\emph{Plan 9}汇编和C语言实现了启动引导程序(bootloader),串口驱动和显示器LCD驱动
% 		%	\item Compiled the OS kernel using the ARM cross compiler and loaded it into the memory according to the memory model of S3C6410
% 		%	\item Debugged the code and finished the porting program successfully
% 	\end{itemize}
% }

%----------------------------------------------------------------------------------------
% TEACHING SECTION
%----------------------------------------------------------------------------------------
\section{助教经历}
\cvitem{}{CMSC5735:高级云计算专题 \hfill 2018春,\textbf{香港中文大学}}
\cvitem{}{CSCI3320:机器学习之基础课程 \hfill 2018春,\textbf{香港中文大学}}
\cvitem{}{CSCI2040:Python程序语言导论 \hfill 2017秋,\textbf{香港中文大学}}

%----------------------------------------------------------------------------------------
%	ACHIEVEMENTS SECTION
%----------------------------------------------------------------------------------------

%\section{EXTRACURRICULAR ACTIVITIES}
%
%\cvitem{Sep.2013--Jun.2015}{Minister, Department of Sports, Student Union}
%\cvitem{Sep.2013--Jun.2017}{Member of Linux User Group, USTC}
%\cvitem{Oct.2013}{Volunteer of Special Education Center in Hefei}

%----------------------------------------------------------------------------------------
%	AWARDS SECTION
%----------------------------------------------------------------------------------------

\section{获奖经历}
\cvitem{}{优秀学生奖学金银奖(前10\%),两次 \hfill 2015--2016, \textbf{中国科学技术大学}}
\cvitem{}{英才班拔尖计划助学金A类(前10\%),三次 \hfill 2014--2016, \textbf{中国科学技术大学}}
\cvitem{}{优秀学生干部(前3\%) \hfill 2015, \textbf{中国科学技术大学}}
\cvitem{}{优秀学生奖学金铜奖(前15\%) \hfill 2014, \textbf{中国科学技术大学}}


%----------------------------------------------------------------------------------------
%	COMPUTER SKILLS SECTION
%----------------------------------------------------------------------------------------

\section{计算机技术}
\cvitemwithcomment{操作系统}{Gentoo Linux, macOS}{}
\cvitemwithcomment{编程语言}{熟悉C/C++, Rust, Python}{}
\cvitemwithcomment{编辑器}{Emacs, Vim}{}


%----------------------------------------------------------------------------------------
%	COMMUNICATION SKILLS SECTION
%----------------------------------------------------------------------------------------

%\section{Communication Skills}

%\cvitem{2010}{Oral Presentation at the California Business Conference}
%\cvitem{2009}{Poster at the Annual Business Conference in Oregon}

%----------------------------------------------------------------------------------------
%	LANGUAGES SECTION
%----------------------------------------------------------------------------------------

%\section{Languages}

%\cvitemwithcomment{English}{Mothertongue}{}
%\cvitemwithcomment{Spanish}{Intermediate}{Conversationally fluent}
%\cvitemwithcomment{Dutch}{Basic}{Basic words and phrases only}

%----------------------------------------------------------------------------------------
%	INTERESTS SECTION
%----------------------------------------------------------------------------------------

%\section{Interests}

%\renewcommand{\listitemsymbol}{-~} % Changes the symbol used for lists

%\cvlistdoubleitem{Piano}{Chess}
%\cvlistdoubleitem{Cooking}{Dancing}
%\cvlistitem{Running}

%----------------------------------------------------------------------------------------

\end{document}

%%% Local Variables:
%%% coding: utf-8
%%% mode: latex
%%% TeX-master: t
%%% TeX-engine: xetex
%%% End:
