% !TEX program = xelatex
%%%%%%%%%%%%%%%%%%%%%%%%%%%%%%%%%%%%%%%%%
% "ModernCV" CV and Cover Letter
% LaTeX Template
% Version 1.3 (29/10/16)
%
% This template has been downloaded from:
% http://www.LaTeXTemplates.com
%
% Original author:
% Xavier Danaux (xdanaux@gmail.com) with modifications by:
% Vel (vel@latextemplates.com)
%
% License:
% CC BY-NC-SA 3.0 (http://creativecommons.org/licenses/by-nc-sa/3.0/)
%
% Important note:
% This template requires the moderncv.cls and .sty files to be in the same 
% directory as this .tex file. These files provide the resume style and themes 
% used for structuring the document.
%
%%%%%%%%%%%%%%%%%%%%%%%%%%%%%%%%%%%%%%%%%

%----------------------------------------------------------------------------------------
%	PACKAGES AND OTHER DOCUMENT CONFIGURATIONS
%----------------------------------------------------------------------------------------

\documentclass[11pt,a4paper,roman]{moderncv} % Font sizes: 10, 11, or 12; paper sizes: a4paper, letterpaper, a5paper, legalpaper, executivepaper or landscape; font families: sans or roman

\moderncvstyle{banking} % CV theme - options include: 'casual' (default), 'classic', 'oldstyle' and 'banking'
\moderncvcolor{black} % CV color - options include: 'blue' (default), 'orange', 'green', 'red', 'purple', 'grey' and 'black'

\usepackage{lipsum} % Used for inserting dummy 'Lorem ipsum' text into the template
%\usepackage{fontspec}
\usepackage{fontawesome}
\usepackage[scale=0.85]{geometry} % Reduce document margins
%\setlength{\hintscolumnwidth}{3cm} % Uncomment to change the width of the dates column
%\setlength{\makecvtitlenamewidth}{10cm} % For the 'classic' style, uncomment to adjust the width of the space allocated to your name
\usepackage{fontspec}
\usepackage{xunicode}
\usepackage{xeCJK} % Chinese font setting
% \setCJKmainfont{Source Han Sans CN}
% \setCJKsansfont{Source Han Sans CN}

\setCJKmainfont[
BoldFont=Heiti SC,
ItalicFont=KaiTi,
SmallCapsFont=SimHei
]{SimSun}
\setCJKsansfont{KaiTi}
\setCJKmonofont{SimHei}

%----------------------------------------------------------------------------------------
%	NAME AND CONTACT INFORMATION SECTION
%----------------------------------------------------------------------------------------
\firstname{李} % Your first name
\familyname{卓华} % Your last name

% All information in this block is optional, comment out any lines you don't need
%\title{Curriculum Vitae}
%\title{Resume}
%\address{The Chinese University of Hong Kong, Shatin, N.T., Hong Kong SAR}{}
\phone[mobile]{(+852)~6587~3796}
%\phone{(000) 111 1112}
%\fax{(000) 111 1113}
\email{zhli@cse.cuhk.edu.hk}
\homepage{zhuohua.me}
\social[github]{lizhuohua}
\social[linkedin]{lizhuohua}
%\extrainfo{additional information}
%\photo[70pt][0.4pt]{pictures/picture} % The first bracket is the picture height, the second is the thickness of the frame around the picture (0pt for no frame)
%\quote{"A witty and playful quotation" - John Smith}

%----------------------------------------------------------------------------------------

\begin{document}

%----------------------------------------------------------------------------------------
%	COVER LETTER
%----------------------------------------------------------------------------------------

% To remove the cover letter, comment out this entire block

%\clearpage

%\recipient{HR Department}{Corporation\\123 Pleasant Lane\\12345 City, State} % Letter recipient
%\date{\today} % Letter date
%\opening{Dear Sir or Madam,} % Opening greeting
%\closing{Sincerely yours,} % Closing phrase
%\enclosure[Attached]{curriculum vit\ae{}} % List of enclosed documents

%\makelettertitle % Print letter title

%\lipsum[1-2] % Dummy text
%\lipsum[4] % Dummy text

%\makeletterclosing % Print letter signature

%\newpage

%----------------------------------------------------------------------------------------
%	CURRICULUM VITAE
%----------------------------------------------------------------------------------------
\vspace*{-5mm}
\makecvtitle % Print the CV title

%----------------------------------------------------------------------------------------
%	EDUCATION SECTION
%----------------------------------------------------------------------------------------
\vspace*{-10mm}
\section{教育经历}

\cventry{2017年8月--现在}{博士,计算机科学与工程学系}{香港中文大学}{香港}{}{网络系统研究实验室(ANSR Lab),导师:John C.S. Lui教授}
%\cventry{Jul.2016--Sep.2016}{Research Intern at Social Computing Group}{Microsoft Research Asia}{Beijing}{}{Mentor: Dr. Xing Xie}  % Arguments not required can be left empty
%\cventry{Aug.2013--Jun.2017}{B.E. in Computer Science and Technology}{University of Science and Technology of China}{Hefei}{\textit{GPA: 3.73/4.3}}{Hua Xia Talent Program in Computer Science and Technology}
\cventry{2013年8月--2017年6月}{学士,计算机科学与技术系}{中国科学技术大学}{合肥}{}{华夏计算机科技英才班}
%Thesis Title: Analysis and Mitigation of Ret2art Attack for Android
%\section{Masters Thesis}

%\cvitem{Title}{\emph{Money Is The Root Of All Evil -- Or Is It?}}
%\cvitem{Supervisors}{Professor James Smith \& Associate Professor Jane Smith}
%\cvitem{Description}{This thesis explored the idea that money has been the cause of untold anguish and suffering in the world. I found that it has, in fact, not.}

%----------------------------------------------------------------------------------------
%	WORK EXPERIENCE SECTION
%----------------------------------------------------------------------------------------

\section{研究经历}

%\subsection{Vocational}
\cventry{2019年7月--2021年5月}{导师:John C.S. Lui\ 教授}{Rust程序的静态分析}{香港中文大学}{}
{
	\begin{itemize}
    \item 基于\textit{抽象释义}开发了针对Rust语言的静态分析和除错工具
		\item 分析运行于Rust的\textit{中级中间表示}(\textit{Mid-level Intermediate Representation})
		\item 检测到了数个现实中的漏洞
	\end{itemize}
}

\cventry{2018年7月--2018年10月}{导师:韦韬\ 博士}{用Rust开发安全的Linux内核模块}{百度X-Lab,美国}{}
{
	\begin{itemize}
		\item 开发了用Rust编写Linux内核模块的框架
		\item 提供了安全的内核态内存管理机制和内核锁同步机制
	\end{itemize}
}

\cventry{2016年7月--2016年9月}{导师:谢幸\ 博士}{微软小冰的情感分析}{微软亚洲研究院,北京}{}
{
    \begin{itemize}
    	\item 参与改进微软小冰聊天机器人的情感分析能力
%	\item Wrote a crawler with \emph{C$^\#$} to crawl sentences from Weibo, tagged their emotions according to their emoticons, extracted a training set and generate a classifier
	    \item 根据对话的上下文识别当前对话中用户的情感,并作出相应回答
%	\item Proposed to simplify the classification of sentences by actual application scenarios and improved the precision of emotion recognition for Xiao Ice
	\end{itemize}
}

\cventry{2016年6月}{导师:安虹\ 教授}{ISC16国际大学生超级计算机竞赛}{德国,法兰克福}{}{
\begin{itemize}
\item 基于华为\emph{FusionServer X6800}服务器搭建超算集群
\item 在3千瓦功耗限制下优化应用性能
%\item Took charge of optimizing \emph{WRF (Weather Research and Forecasting Model)}
\item 获得总分第四名,\emph{WRF (Weather Research and Forecasting Model)}单项第一名
%\begin{itemize}
%\item Wrote shell scripts to make the time-consuming compiling and linking done automatically
%\item Modified the compiling options to be depth optimization of the program
%\item Analyzed the operational characteristics of \emph{WRF} with \emph{Intel\textsuperscript{\textregistered}\ VTune\texttrademark\ Amplifier}, \emph{Paramon / Paratune} to find the hardware bottleneck and did the corresponding upgrade
%\item Replaced the \emph{FFT(Fast Fourier Transform)} module with \emph{Intel\textsuperscript{\textregistered}\ Math Kernel Library (MKL)} and tried different \emph{MPI}, \emph{OpenMP} parameters
%\item Found out the optimal number of processes and threads which led us to win the \textbf{1\textsuperscript{st}} place for \emph{WRF}
%\end{itemize}
\end{itemize}
}


%\cventry{2012--Present}{1\textsuperscript{st} Year Analyst}{\textsc{Lehman Brothers}}{Los Angeles}{}{Developed spreadsheets for risk analysis on exotic derivatives on a wide array of commodities (ags, oils, precious and base metals), managed blotter and secondary trades on structured notes, liaised with Middle Office, Sales and Structuring for bookkeeping.
%\newline{}\newline{}
%Detailed achievements:
%\begin{itemize}
%\item Learned how to make amazing coffee
%\item Finally determined the reason for \textsc{PC LOAD LETTER}:
%\begin{itemize}
%\item Paper jam
%\item Software issues:
%\begin{itemize}
%\item Word not sending the correct data to printer
%\item Windows trying to print in letter format
%\end{itemize}
%\item Coffee spilled inside printer
%\end{itemize}
%\item Broke the office record for number of kitten pictures in cubicle
%\end{itemize}}

%------------------------------------------------

%\cventry{2011--2012}{Summer Intern}{\textsc{Lehman Brothers}}{Los Angeles}{}{Rated "truly distinctive" for Analytical Skills and Teamwork.}

%------------------------------------------------

%\subsection{Miscellaneous}

%\cventry{2010--2011}{}{}{}{}{Spent some time finding myself. This was a courageous endeavour that didn't have a job title. It was quite important to my overall development though so I'm adding it to my CV. Also it explains the gap in my otherwise stellar CV.}

%\cventry{2009--2010}{Computer Repair Specialist}{Buy More}{Burbank}{}{Worked in the Nerd Herd and helped to solve computer problems. Allowed me to become expert in all forms of martial arts and weaponry.}

\section{发表论文}
\cvitem{}{\textbf{Zhuohua Li}, Jincheng Wang, Mingshen Sun, and John C.S. Lui. MirChecker: Detecting Bugs in Rust Programs via Static Analysis. In \textit{Proceedings of the 28th ACM Conference on Computer and Communications Security}. CCS '21.}
\cvitem{}{Jincheng Wang, \textbf{Zhuohua Li}, John C.S. Lui, and Mingshen Sun. Topology-Theoretic Approach To Address Attribute Linkage Attacks In Differential Privacy. Computers \& Security (2021): 102552.}
\cvitem{}{\textbf{Zhuohua Li}, Jincheng Wang, Mingshen Sun, and John C.S. Lui. Securing the Device Drivers of Your Embedded Systems: Framework and Prototype. In \textit{Proceedings of the 3rd International Workshop on Security and Forensics of IoT (in conjunction with ARES 2019)}. IoT-SECFOR '19.}

\section{助教}
\cvitem{}{CMSC5735:高级云计算专题 \hfill \textbf{2018春,香港中文大学}}
\cvitem{}{CSCI3320:机器学习之基础课程 \hfill \textbf{2018春,香港中文大学}}
\cvitem{}{CSCI2040:Python程序语言导论 \hfill \textbf{2017秋,香港中文大学}}

%----------------------------------------------------------------------------------------
%	Projects SECTION
%----------------------------------------------------------------------------------------

\section{项目}

\cventry{2015年9月--2016年2月}{导师:张煜\ 副教授}{基于LLVM的简易C语言编译器}{中国科学技术大学}{}
{
	\begin{itemize}
		\item 使用\emph{Flex}+\emph{Bison}+\emph{LLVM}实现的简易C语言编译器
		\item 支持Clang风格的报错信息
		\item 支持编译基于\textit{Message Passing Interface (MPI)}的并行程序
	\end{itemize}
}

\cventry{2015年3月--2015年7月}{导师:李希\ 副教授}{基于FPGA实现的MIPS16e处理器}{中国科学技术大学}{}
{
	\begin{itemize}
    \item 在\emph{Xilinx FPGA}上使用\emph{Verilog HDL}语言实现,支持全部\emph{MIPS16e}指令
		\item 通过\emph{RS-232}串口实现DEBUG模块,可打印DEBUG信息到屏幕 
	\end{itemize}
}

\cventry{2015年3月--2015年6月}{导师:邢凯\ 副教授}{移植Inferno操作系统至三星S3C6410开发板}{中国科学技术大学}{}
{
	\begin{itemize}
		\item 使用\emph{Plan 9}汇编和C语言实现了启动引导程序(bootloader),串口驱动和显示器LCD驱动
		%	\item Compiled the OS kernel using the ARM cross compiler and loaded it into the memory according to the memory model of S3C6410
		%	\item Debugged the code and finished the porting program successfully
	\end{itemize}
}

%----------------------------------------------------------------------------------------
%	ACheivements SECTION
%----------------------------------------------------------------------------------------

%\section{EXTRACURRICULAR ACTIVITIES}
%
%\cvitem{Sep.2013--Jun.2015}{Minister, Department of Sports, Student Union}
%\cvitem{Sep.2013--Jun.2017}{Member of Linux User Group, USTC}
%\cvitem{Oct.2013}{Volunteer of Special Education Center in Hefei}

%----------------------------------------------------------------------------------------
%	AWARDS SECTION
%----------------------------------------------------------------------------------------

\section{获奖经历}

\cvitem{}{优秀学生奖学金银奖(前10\%),两次 \hfill 2015--2016, \textbf{中国科学技术大学}}
\cvitem{}{英才班拔尖计划助学金A类(前10\%),三次 \hfill 2014--2016, \textbf{中国科学技术大学}}
\cvitem{}{优秀学生干部(前3\%) \hfill 2015, \textbf{中国科学技术大学}}
\cvitem{}{优秀学生奖学金铜奖(前15\%) \hfill 2014, \textbf{中国科学技术大学}}


%----------------------------------------------------------------------------------------
%	COMPUTER SKILLS SECTION
%----------------------------------------------------------------------------------------

\section{计算机技术}

\cvitemwithcomment{操作系统}{Gentoo Linux, macOS}{}
\cvitemwithcomment{编程语言}{熟悉C/C++, Rust, Python;了解OCaml, Haskell}{}
\cvitemwithcomment{编辑器}{Emacs, Vim}{}


%----------------------------------------------------------------------------------------
%	COMMUNICATION SKILLS SECTION
%----------------------------------------------------------------------------------------

%\section{Communication Skills}

%\cvitem{2010}{Oral Presentation at the California Business Conference}
%\cvitem{2009}{Poster at the Annual Business Conference in Oregon}

%----------------------------------------------------------------------------------------
%	LANGUAGES SECTION
%----------------------------------------------------------------------------------------

%\section{Languages}

%\cvitemwithcomment{English}{Mothertongue}{}
%\cvitemwithcomment{Spanish}{Intermediate}{Conversationally fluent}
%\cvitemwithcomment{Dutch}{Basic}{Basic words and phrases only}

%----------------------------------------------------------------------------------------
%	INTERESTS SECTION
%----------------------------------------------------------------------------------------

%\section{Interests}

%\renewcommand{\listitemsymbol}{-~} % Changes the symbol used for lists

%\cvlistdoubleitem{Piano}{Chess}
%\cvlistdoubleitem{Cooking}{Dancing}
%\cvlistitem{Running}

%----------------------------------------------------------------------------------------

\end{document}

%%% Local Variables:
%%% coding: utf-8
%%% mode: latex
%%% TeX-master: t
%%% TeX-engine: xetex
%%% End: